\documentclass{jarticle}

\title{計算機システム演習 第三回レポート}
\author{17B13541 \and 細木隆豊}
\date{}

\begin{document}
\maketitle

  \section{説明・工夫}
    \subsection{sltを用いたbltの実装(assigenment1blt.s)}

    sltを用いない場合
    \begin{center}
    blt   \$s1, \$s2, label
    \end{center}
    と記述するが、sltを用いた場合
    \begin{center}
    slt   \$t0, \$s1, \$s2\\
    bne   \$t0, \$zero, label
    \end{center}
    となる。

    \$s1 $<$ \$s2 を満たすとき、labelへjumpすればよい。\\
    sltを用いると、条件を満たすとき \$t0 $=$ 1 となる。0 と \$t0 を比較し、not equal ならば j~~label よりbneを用いた。\\
    比較で\$zeroを用いると、新たにレジスタに値をいれなくてよいのでコードが短くなる。
    \subsection{sltを用いたbleの実装}
  \section{実行結果}
  \section{感想}
\end{document}
